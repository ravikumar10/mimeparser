%% Tu mo�na zdefiniowa� w�asne komendy:
\newcommand{\ang}[1]{(ang. {\em #1}\/)}
\newcommand{\e}[1]{{\em #1}\/}

\newcommand{\strona}[1]{(str. \pageref{#1})}
\newcommand{\punkt}[1]{(rys. \ref{#1})}
\newcommand{\rys}[1]{(rys. \ref{#1})}

%% styl do malowania w ramkach polecen SMTP normalny tryb
\newcommand{\combox}[1]{
\newline
\newline
\framebox[0.95\textwidth][l]{
\begin{minipage}[c]{14cm}
\textit{#1}
\end{minipage}
}
\newline
\newline
}

%% styl do malowania w ramkach polecen SMTP bez 
%% dwoch koncowych \newlineow
\newcommand{\comboxnonl}[1]{
\newline
\newline
\framebox[0.95\textwidth][l]{
\begin{minipage}[c]{14cm}
\textit{#1}
\end{minipage}
}
}


%% styl do malowania w ramkach polecen SMTP w itemize
\newcommand{\comboxinitemize}[1]{
\newline
\newline
\framebox[0.90\textwidth][l]{
\begin{minipage}[c]{14cm}
\textit{#1}
\end{minipage}
}
\newline
\newline
}

%% styl do malowania ramek zeby bo nich byly ladne
%% odstepy
\newcommand{\newlines}[1]{
\newline
\newline
\makebox[0.95\textwidth][l]{
\begin{minipage}[c]{14cm}
#1
\end{minipage}
}
\newline
\newline
\newline
}

%% styl do malowania w ramkach polecen SMTP w itemize
%% bez koncow lini
\newcommand{\comboxinitemizenonl}[1]{
\framebox[0.90\textwidth][l]{
\begin{minipage}[c]{13cm}
\textit{#1}
\end{minipage}
}
}

%% styl do malowania couriere polecen
\newcommand{\comcourier}[1] {
#1
}